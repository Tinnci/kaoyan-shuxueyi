% !TEX program = xelatex
\documentclass[UTF8,a4paper,12pt]{ctexart}

% 版式与间距
\usepackage[margin=2.2cm]{geometry}
\usepackage{setspace}
\onehalfspacing

% 字体
\usepackage{fontspec}
\setmainfont{Times New Roman}
\setsansfont{Arial}
\setmonofont{Consolas}

% 数学
\usepackage{amsmath, amssymb, amsfonts, mathtools}
\allowdisplaybreaks[1]

% 颜色/盒子/标题样式
\usepackage[dvipsnames]{xcolor}
\usepackage[most]{tcolorbox}
\tcbset{colback=white,colframe=MidnightBlue!60!black,boxrule=0.6pt,arc=2pt,left=6pt,right=6pt,top=6pt,bottom=6pt}
\usepackage{titlesec}
\titleformat{\section}{\Large\bfseries\color{MidnightBlue}}{\thesection}{0.5em}{}[\titlerule]
\titleformat{\subsection}{\large\bfseries\color{MidnightBlue}}{}{0pt}{}
\titlespacing*{\section}{0pt}{0.8em}{0.5em}
\titlespacing*{\subsection}{0pt}{0.6em}{0.4em}
\usepackage{enumitem}
\setlist[itemize]{noitemsep, topsep=2pt, leftmargin=1.2em}

% 链接
\usepackage[colorlinks=true,linkcolor=blue,citecolor=blue,urlcolor=blue]{hyperref}

% 图形
\usepackage{tikz}
\usetikzlibrary{arrows.meta,calc,angles,quotes,intersections}

% 页眉页脚
\usepackage{fancyhdr}
\pagestyle{fancy}
\fancyhf{}
\fancyhead[L]{旋转体体积·点到直线距离记忆法}
\fancyhead[R]{\today}
\fancyfoot[C]{\thepage}
\renewcommand{\headrulewidth}{0.4pt}
\renewcommand{\footrulewidth}{0pt}

% 标题
\usepackage{titling}
\pretitle{\begin{center}\begin{tcolorbox}[colback=MidnightBlue!3,colframe=MidnightBlue!60,boxrule=0.8pt,arc=2pt]\LARGE\bfseries}
\posttitle{\\[2pt]\normalsize 背诵逻辑 + 可视化速查\end{tcolorbox}\end{center}}
\title{任意直线旋转体体积的“理解式”背诵}
\author{摘要整理}
\date{\today}

\begin{document}
\maketitle

\section{引子}
\begin{tcolorbox}
你好,我们来看这个公式。\\[4pt]
这个公式确实又长又复杂,想要死记硬背它几乎是不可能的,而且很容易出错。\textbf{对付这种复杂公式,最好的“背诵”方法是理解它每个部分的含义},把它看成一个故事,而不是一串随机的符号。
\end{tcolorbox}

我们从最熟悉的开始,一步步把它“造”出来。

\section{第一步:核心思想}
所有旋转体体积公式,核心思想都是一样的:把图形切成无数个薄片(通常是圆盘),算出每个薄片的体积,然后用积分加起来。
\begin{tcolorbox}
一个薄片的体积 = 底面积 $\times$ 厚度 = $\pi r^2 \times (\text{厚度})$,因此总体积的基本框架:
\begin{equation}
V = \int \pi\, (\text{半径})^2\, d(\text{厚度}).
\end{equation}
\end{tcolorbox}

\section{第二步:确定“半径”}
\begin{itemize}
  \item 绕 \(x\) 轴(\(y=0\))旋转时,半径就是 \(r=f(x)\)。
  \item 绕一般直线 \(Ax+By+C=0\) 旋转时,半径 \(r\) 是曲线上点 \((x,f(x))\) 到该直线的\textbf{垂直距离}。
\end{itemize}
点 \((x_0,y_0)\) 到直线 \(Ax+By+C=0\) 的距离公式:
\begin{equation}
 d=\frac{|Ax_0+By_0+C|}{\sqrt{A^2+B^2}}.
\end{equation}
将 \((x_0,y_0)=(x,f(x))\) 代入,可得半径的平方
\begin{tcolorbox}
\begin{equation}
 r^2=\left(\frac{|Ax+Bf(x)+C|}{\sqrt{A^2+B^2}}\right)^2=\frac{\big[A x + B f(x) + C\big]^2}{A^2+B^2}.
\end{equation}
\end{tcolorbox}
这就是公式中最核心、最“复杂”的那一部分。

\subsection*{可视化:点到直线的垂距}
\begin{center}
\begin{tikzpicture}[scale=1.0]
  % 坐标轴(淡色)
  \draw[->,gray!40] (-0.5,0) -- (6.2,0) node[below]{\small $x$};
  \draw[->,gray!40] (0,-0.5) -- (0,4.2) node[left]{\small $y$};
  % 任意直线 Ax+By+C=0 的一条实例(画成斜线)
  \draw[thick,MidnightBlue] (-0.2,2.0) -- (6.0,0.8) node[below right]{\small $Ax+By+C=0$};
  % 曲线 y=f(x)(示意为一条平滑曲线)
  \draw[smooth,thick,ForestGreen,domain=0.3:5.5,samples=60] plot(\x,{0.2*(\x-1)*(\x-4)+2.2}) node[right]{\small $y=f(x)$};
  % 取曲线上一点 P
  \def\xx{2.6}
  \pgfmathsetmacro{\yy}{0.2*(\xx-1)*(\xx-4)+2.2}
  \fill[red] (\xx,\yy) circle (2pt) node[above left]{$P(x,f(x))$};
  % 作到直线的垂线
  % 直线方向向量与法向量
  \coordinate (P) at (\xx,\yy);
  \coordinate (A) at (0,2);
  \coordinate (B) at (6,0.8);
  % 计算法向量 n(由 AB 方向向量旋转得到)
  \path (A);
  % 通过投影构造垂足 Q(确保为“长线”以保证相交)
  \draw[name path=lineAB,MidnightBlue] ($(A)!-2!(B)$) -- ($(A)!3!(B)$);
  % 过 P 作与 AB 垂直的长直线(AB 方向向量为 (6,-1.2),其垂直向量可取 (1.2,6))
  \draw[name path=perp,transparent] ($(P)+(-1.2,-6)$) -- ($(P)+(1.2,6)$);
  \path[name intersections={of=lineAB and perp, by=Q}];
  \draw[red,thick] (P) -- (Q) node[midway,right]{\small $r$};
  \fill[black] (Q) circle (1.3pt) node[below left]{\small $Q$};
  % 垂直符号
  \pic [draw=red, angle radius=5pt, angle eccentricity=1.3] {right angle = B--Q--P};
\end{tikzpicture}
\end{center}

\section{第三步:确定“厚度”}
\begin{itemize}
  \item 绕 \(x\) 轴旋转:薄片为竖直圆盘,厚度就是 \(dx\)。
  \item 绕斜线旋转:薄片相对轴线倾斜,“厚度”不再是简单的 \(dx\),需要一个\textbf{修正因子}(由几何关系或坐标变换给出)。
\end{itemize}

\section{把三部分合在一起}
\begin{tcolorbox}
\textbf{基本框架}:
\begin{equation}
V = \pi \int_{x=a}^{b} \big(\text{半径}~r(x)\big)^2\, d(\text{厚度}).
\end{equation}
\textbf{核心内容(半径平方)} 来自点到直线距离:
\begin{equation}
 r^2(x)=\frac{\big[A x + B f(x) + C\big]^2}{A^2+B^2}.
\end{equation}
\textbf{修正项}:当旋转轴是斜线时,引入适当的几何修正(可通过坐标旋转/平移把轴化为新坐标系的 $\tilde x$ 轴,再使用 $d\tilde x$)。
\end{tcolorbox}

\subsection*{书上公式(带修正因子)}
将上述三部分合并,在 $x\in[a,b]$ 上,书上公式可写为
\begin{tcolorbox}
\begin{equation}\label{eq:book}
\boxed{\;V\;=\;\frac{\pi}{\big(A^2+B^2\big)^{3/2}}\,\int_{a}^{b} \Big(Ax+Bf(x)+C\Big)^{\!2}\,\big|A f'(x)-B\big|\,dx\;}
\end{equation}
\end{tcolorbox}
这里 $Ax+By+C=0$ 为旋转轴,\,$y=f(x)$ 为被旋转曲线。分母的 $\big(A^2+B^2\big)^{3/2}$ 与被积中的 $\big|A f'(x)-B\big|$ 来自“厚度”方向与坐标旋转的几何修正。

\subsection*{修正因子的来由(坐标旋转速推)}
设 $\theta$ 满足 $\cos\theta=\dfrac{A}{\sqrt{A^2+B^2}},\;\sin\theta=\dfrac{B}{\sqrt{A^2+B^2}}$。以该 $\theta$ 旋转坐标,使轴线 $Ax+By+C=0$ 变为新坐标中的水平线 $\tilde y=\text{const}$,则有
\begin{align}
\tilde x 
  &= x\cos\theta + y\sin\theta 
   = \frac{Ax+By}{\sqrt{A^2+B^2}},\\
\tilde y 
  &= -x\sin\theta + y\cos\theta 
   = \frac{-Bx+Ay}{\sqrt{A^2+B^2}}.
\end{align}
代入 $y=f(x)$ 并微分:$d\tilde x 
= \dfrac{A+ B f'(x)}{\sqrt{A^2+B^2}}\,dx$,而沿法向(到轴线的最短距离)对应的厚度元与 $\tilde x$ 的变化成比例。将半径取为到轴线的垂距
$\;r(x)= \dfrac{|Ax+Bf(x)+C|}{\sqrt{A^2+B^2}}\;$,圆盘面积为 $\pi r^2$。综合几何关系可得到式\,(\ref{eq:book}) 中的整体修正因子
$\big(A^2+B^2\big)^{-3/2}$ 与 $\big|A f'(x)-B\big|$(符号来自法向与切向的相对方向,取绝对值以保证体积为正)。

\noindent 提示:你也可以通过将变量沿轴向和法向分解(雅可比行列式)来得到相同的结果。

\subsection*{对比:特殊情形(绕 $x$ 轴)}
当旋转轴是 \(y=0\) 时,\(A=0,B=1,C=0\),于是
\begin{equation}
 r(x)=|f(x)|,\qquad r^2(x)=f(x)^2,\qquad V=\pi\int_a^b f(x)^2\,dx.
\end{equation}

\section{总结:如何“记住”它}
你可以把这个公式看成三个部分的组合:
\begin{enumerate}[leftmargin=1.4em]
  \item \textbf{基本框架:} $V=\pi\int_a^b (\cdots)\,dx$。
  \item \textbf{核心内容(半径平方):} 来源于点到直线距离公式,$\dfrac{[Ax+Bf(x)+C]^2}{A^2+B^2}$。
  \item \textbf{修正项:} 因轴线倾斜,需要厚度方向的修正;常通过\emph{坐标旋转/平移}化简。
\end{enumerate}
因此记忆要点是:\textbf{体积是 $\pi$ 乘以“点到直线距离的平方”的积分,再根据具体情况进行修正}。下次遇到类似题,可以先写出点到直线的距离,再考虑厚度的修正,这样就不必死记硬背。

\subsection*{额外的可视化:旋转薄片的示意}
\begin{center}
\begin{tikzpicture}[scale=1]
  % 轴线 l
  \draw[thick,MidnightBlue] (-0.5,0.5) -- (6.2,1.6) node[below right]{\small 旋转轴 $Ax+By+C=0$};
  % 取一点 P 与半径 r
  \coordinate (P2) at (2.8,2.3);
  \draw[red,thick] (P2) -- ++(-0.86,-0.18) coordinate (Q2);
  \fill[red] (P2) circle (2pt) node[above left]{\small $P$};
  \fill[black] (Q2) circle (1.3pt) node[below]{\small $Q$};
  \node at ($(P2)!0.5!(Q2)+(0.1,-0.1)$) {\small $r$};
  % 画一个与轴线垂直的圆盘(薄片)
  \begin{scope}[shift={(Q2)}]
    % 局部坐标系:使圆盘法向与轴线重合
    \draw[fill=blue!10,draw=blue!60] (0,0) circle [x radius=0.9cm, y radius=0.25cm];
  \end{scope}
  \node[blue!60] at ($(Q2)+(1.2,0.4)$) {\small 圆盘面积 $\approx \pi r^2$};
  % 厚度方向箭头(示意 d(厚度))
  \draw[->,gray!60] (4.2,2.0) -- (5.0,2.15) node[right]{\small $d(\text{厚度})$};
\end{tikzpicture}
\end{center}

\vspace{0.6em}
\begin{tcolorbox}
\textbf{速查卡:}
\begin{align}
&\text{点到直线距离:}\quad d=\frac{|Ax_0+By_0+C|}{\sqrt{A^2+B^2}}.\\
&\text{半径平方(代入 }(x,f(x))\text{):}\quad r^2=\frac{\big[A x + B f(x) + C\big]^2}{A^2+B^2}.\\
&\text{体积分解:}\quad V=\int \pi r^2\, d(\text{厚度}).
\end{align}
\end{tcolorbox}

\end{document}

